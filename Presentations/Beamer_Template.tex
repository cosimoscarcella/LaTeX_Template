%%%%%%%%%%%%%%%%%%%%%%%%%%%%%%%%%%%%%%%%%
% Beamer Presentation Template
%
% Author Cosimo Damiano Scarcella
% Date -
%
%%%%%%%%%%%%%%%%%%%%%%%%%%%%%%%%%%%%%%%%%

%----------------------------------------------------------------------------------------
%	PACKAGES AND THEMES
%----------------------------------------------------------------------------------------

\documentclass[xcolor=table]{beamer}
%\documentclass[handout]{beamer} %Product printable version of presentation

\usepackage[utf8]{inputenc}
\usepackage[italian]{babel}
\usepackage{amsmath}
\usepackage{amsfonts}
\usepackage{amssymb}
\usepackage{hyperref}
\usepackage{graphicx}
\usepackage[table]{xcolor}
\usepackage{wrapfig}
\usepackage{float}
\usepackage{caption}
\usepackage{subcaption}
\usepackage{listings}
\usepackage{indentfirst}
\usepackage{pdfpages}
\usepackage{fancyhdr}
\usepackage{calc}
\usepackage{setspace}
\usepackage{fixltx2e}
\usepackage{multicol}
\usepackage[normalem]{ulem}
\usepackage{microtype}     % microtypography, reduces hyphenation}
\linespread{1.2}
%\pagestyle{fancy}
\frenchspacing

% Color Definition
\definecolor{lightgray}{rgb}{0.83, 0.83, 0.83}
\definecolor{lightblue}{rgb}{0.68, 0.85, 0.9}
\definecolor{lightgreen}{rgb}{0.56, 0.93, 0.56}
\definecolor{lightcyan}{rgb}{0.88, 1.0, 1.0}
\definecolor{lightbrown}{rgb}{0.71, 0.4, 0.11}

\lstset{%
  backgroundcolor=\color{lightgray},   	% choose the background color; you must add \usepackage{color} or \usepackage{xcolor}
  basicstyle=\scriptsize\ttfamily,   	% the size of the fonts that are used for the code
  breakatwhitespace=false,         		% sets if automatic breaks should only happen at whitespace
  breaklines=true,                		% sets automatic line breaking
  captionpos=b,                    		% sets the caption-position to bottom
  commentstyle=\color{black},    		% comment style
  %deletekeywords={...},            	% if you want to delete keywords from the given language
  %escapeinside={\%*}{*)},          	% if you want to add LaTeX within your code
  extendedchars=true,              		% lets you use non-ASCII characters; for 8-bits encodings only, does not work with UTF-8
  %frame=L,                    			% adds a frame around the code
  %identifierstyle=\color{black},
  keepspaces=true,                 		% keeps spaces in text, useful for keeping indentation of code (possibly needs columns=flexible)
  keywordstyle=\color{black},       	% keyword style
  %language=bash,                 		% the language of the code
  %morekeywords={*,...},            	% if you want to add more keywords to the set
  %numbers=left,                   		% where to put the line-numbers; possible values are (none, left, right)
  %numbersep=1pt,                   	% how far the line-numbers are from the code
  numberstyle=\color{black}, 			% the style that is used for the line-numbers
  rulecolor=\color{black},         		% if not set, the frame-color may be changed on line-breaks within not-black text (e.g. comments (green here))
  showspaces=false,                		% show spaces everywhere adding particular underscores; it overrides 'showstringspaces'
  showstringspaces=false,          		% underline spaces within strings only
  showtabs=false,                  		% show tabs within strings adding particular underscores
  stepnumber=2,                    		% the step between two line-numbers. If it's 1, each line will be numbered
  stringstyle=\color{black},     		% string literal style
  tabsize=2,                       		% sets default tabsize to 2 spaces
  title=\lstname,                   	% show the filename of files included with \lstinputlisting; also try caption instead of title
  xleftmargin=\parindent
}
\hypersetup{
    %bookmarks=true,         				% show bookmarks bar?
    unicode=false,          				% non-Latin characters in Acrobat’s bookmarks
    pdftoolbar=true,       					% show Acrobat’s toolbar?
    pdfmenubar=true,        				% show Acrobat’s menu?
    pdffitwindow=false,     				% window fit to page when opened
    pdfstartview={FitH},    				% fits the width of the page to the window
    pdftitle={MyTitle},    					% title
    pdfauthor={Cosimo Damiano Scarcella},  	% author
    pdfsubject={MySubject},   				% subject of the document
    pdfcreator={Cosimo Damiano Scarcella}, 	% creator of the document
    pdfproducer={Cosimo Damiano Scarcella},	% producer of the document
    %pdfkeywords={keyword1} {key2} {key3}, 	% list of keywords
    pdfnewwindow=true,      				% links in new window
    colorlinks=true,       					% false: boxed links; true: colored links
    linkcolor=black,          				% color of internal links (change box color with linkbordercolor)
    citecolor=black,        				% color of links to bibliography
    filecolor=magenta,      				% color of file links
    urlcolor=black           				% color of external links
}


\mode<presentation> {

% The Beamer class comes with a number of default slide themes
% which change the colors and layouts of slides. Below this is a list
% of all the themes, uncomment each in turn to see what they look like.

%\usetheme{default}
%\usetheme{AnnArbor}
%\usetheme{Antibes}
%\usetheme{Bergen}
%\usetheme{Berkeley}
%\usetheme{Berlin}
%\usetheme{Boadilla}
%\usetheme{CambridgeUS}
%\usetheme{Copenhagen}
%\usetheme{Darmstadt}
%\usetheme{Dresden}
%\usetheme{Frankfurt}
%\usetheme{Goettingen}
%\usetheme{Hannover}
%\usetheme{Ilmenau}
%\usetheme{JuanLesPins}
%\usetheme{Luebeck}
\usetheme{Madrid}
%\usetheme{Malmoe}
%\usetheme{Marburg}
%\usetheme{Montpellier}
%\usetheme{PaloAlto}
%\usetheme{Pittsburgh}
%\usetheme{Rochester}
%\usetheme{Singapore}
%\usetheme{Szeged}
%\usetheme{Warsaw}

% As well as themes, the Beamer class has a number of color themes
% for any slide theme. Uncomment each of these in turn to see how it
% changes the colors of your current slide theme.

%\usecolortheme{albatross}
%\usecolortheme{beaver}
%\usecolortheme{beetle}
%\usecolortheme{crane}
%\usecolortheme{dolphin}
%\usecolortheme{dove}
%\usecolortheme{fly}
%\usecolortheme{lily}
%\usecolortheme{orchid}
%\usecolortheme{rose}
%\usecolortheme{seagull}
%\usecolortheme{seahorse}
%\usecolortheme{whale}
%\usecolortheme{wolverine}

%\setbeamertemplate{footline} % To remove the footer line in all slides uncomment this line
%\setbeamertemplate{footline}[page number] % To replace the footer line in all slides with a simple slide count uncomment this line

%\setbeamertemplate{navigation symbols}{} % To remove the navigation symbols from the bottom of all slides uncomment this line
}

\usepackage{graphicx} % Allows including images
\usepackage{booktabs} % Allows the use of \toprule, \midrule and \bottomrule in tables

%----------------------------------------------------------------------------------------
%	TITLE PAGE
%----------------------------------------------------------------------------------------

\title[Short title]{Full Title of the Talk} % The short title appears at the bottom of every slide, the full title is only on the title page

\author{Cosimo Damiano Scarcella} % Your name
\institute[Unibo] % Your institution as it will appear on the bottom of every slide, may be shorthand to save space
{
Università di Bologna\\ % Your institution for the title page
\medskip
%\textit{cosimo.scarcella@gmail.com} % Your email address

\href{http://www.cosimoscarcella.com}{www.cosimoscarcella.com}


}
%\titlegraphic{\includegraphics[width=30mm]{./Images/logo.png}}
%\logo{\includegraphics[width=15mm]{./Images/logo.png}}
\date{\today} % Date, can be changed to a custom date

\begin{document}

\begin{frame}
\titlepage % Print the title page as the first slide
\end{frame}

\begin{frame}
\frametitle{Overview} % Table of contents slide, comment this block out to remove it
\tableofcontents % Throughout your presentation, if you choose to use \section{} and \subsection{} commands, these will automatically be printed on this slide as an overview of your presentation
\end{frame}

%----------------------------------------------------------------------------------------
%	PRESENTATION SLIDES
%----------------------------------------------------------------------------------------


%------------------------------------------------
\section{Section} % Sections can be created in order to organize your presentation into discrete blocks, all sections and subsections are automatically printed in the table of contents as an overview of the talk
%------------------------------------------------

\subsection{Subsection} % A subsection can be created just before a set of slides with a common theme to further break down your presentation into chunks

% Cut and Paste frame example 
%------------------------------------------------
%------------------------------------------------
%------------------------------------------------
%------------------------------------------------
%------------------------------------------------
%------------------------------------------------
%------------------------------------------------
%------------------------------------------------
%------------------------------------------------
%------------------------------------------------
%------------------------------------------------
%------------------------------------------------
%------------------------------------------------
%------------------------------------------------
%------------------------------------------------
%------------------------------------------------
%------------------------------------------------
%------------------------------------------------


\begin{frame}
\frametitle{Paragraphs of Text}
text\\~\\

text
\end{frame}

%------------------------------------------------

\begin{frame}
\frametitle{Figure}
hidden image (uncomment)
%\begin{figure}
%\includegraphics[width=0.8\linewidth]{image_path}
%\end{figure}
\end{frame}

%------------------------------------------------

\begin{frame}
\frametitle{Bullet Points}
\begin{itemize}
\item text
\item text
\item text
\item text
\item text
\end{itemize}
\end{frame}

%------------------------------------------------

\begin{frame}
\frametitle{Blocks of Highlighted Text}
\begin{block}{Block 1}
text
\end{block}

\begin{block}{Block 2}
text
\end{block}

\begin{block}{Block 3}
text
\end{block}
\end{frame}

%------------------------------------------------

\begin{frame}
\frametitle{Multiple Columns}
\begin{columns}[c] % The "c" option specifies centered vertical alignment while the "t" option is used for top vertical alignment

\column{.45\textwidth} % Left column and width
\textbf{text}
\begin{enumerate}
\item text
\item text
\item text
\end{enumerate}

\column{.5\textwidth} % Right column and width
text

\end{columns}
\end{frame}

%------------------------------------------------

\begin{frame}
\frametitle{Table 1}
\begin{table}
\begin{tabular}{l l l}
\toprule
\textbf{header1} & \textbf{header2} & \textbf{header3}\\
\midrule
text & text & text \\
text & text & text \\
text & text & text \\
\bottomrule
\end{tabular}
\caption{Table}
\end{table}
\end{frame}

%------------------------------------------------

\begin{frame}
\frametitle{Table 2}
\begin{table}[h]                        
\begin{center}		                    % center table
\rowcolors{1}{white}{lightcyan}			% \rowcolors{<starting row>}{<odd color>}{<even color>}
\begin{tabular}[c]{|l|c|r|}             % three columns with vertical lines
										% [b]: bottom
										% [c]: center (default)
										% [t]: top
										% l: left-justified column
										% c: centered column
										% r: right-justified column
										% |: vertical line
										% ||: double vertical line
										% p{'width'}: paragraph column with text vertically aligned at the top
										% m{'width'}: paragraph column with text vertically aligned in the middle (requires array package)
										% b{'width'}: paragraph column with text vertically aligned at the bottom (requires array package)
										% \newline: start a new line within a cell (in a paragraph column)
										% \cline{i-j}: partial horizontal line beginning in column i and ending in column j

\hline                           		% inserts a horizontal line
\rowcolor{lightblue}
\textbf{header1} & \multicolumn{2}{c|}{\textbf{header2}}\\ % & separating columns
\cline{2-3}
\hline \hline                           % insert two horizontal lines
text & text & text\\          			% \\ start new row
\hline                                  % insert a horizontal line
text & text & text\\          			% \\ start new row
\hline                                  % insert a horizontal line
text & text & \cellcolor{yellow} text\\
\hline \hline                           % insert two horizontal lines

\end{tabular}
\caption{Table Description}
\end{center}
\end{table}
\end{frame}

%------------------------------------------------


\begin{frame}
\frametitle{Theorem}
\begin{theorem}[theorem name]
text
\end{theorem}
\end{frame}

%------------------------------------------------

\begin{frame}[fragile] % Need to use the fragile option when verbatim is used in the slide
\frametitle{Verbatim}
\begin{example}[Code]
\begin{verbatim}
text code
\end{verbatim}
\end{example}
\end{frame}

%------------------------------------------------

% Lstlisting allow language identification for example C, Java, Bash ecc
\begin{frame}[fragile]
\frametitle{Lstlisting}
\begin{example}[Code]
\begin{lstlisting}[language=bash, frame=single]
text code
\end{lstlisting}
\end{example}
\end{frame}

%------------------------------------------------

\begin{frame}[fragile] % Need to use the fragile option when verbatim is used in the slide
\frametitle{Citation}
text\\~

text \cite{p1}.
\end{frame}

%------------------------------------------------

\begin{frame}
\frametitle{References}
\footnotesize{
\begin{thebibliography}{99} % Beamer does not support BibTeX so references must be inserted manually as below
\bibitem[Cosimo, 2015]{p1} Cosimo Damiano Scarcella (2015)
\newblock Title of the publication
\newblock \emph{Journal Name} 12(3), 45 -- 678.
\end{thebibliography}
}
\end{frame}

%------------------------------------------------

\begin{frame}
\Huge{\centerline{The End}}
\end{frame}

%----------------------------------------------------------------------------------------

\end{document} 